\documentclass[12pt]{article}

\usepackage{graphicx}
\usepackage{paralist}
\usepackage{amsfonts}
\usepackage{amsmath}
\usepackage{hhline}
\usepackage{booktabs}
\usepackage{multirow}
\usepackage{multicol}
\usepackage{url}
\usepackage[left = 3cm, right = 3cm, top = 3cm]{geometry}


\pagestyle {plain}
\pagenumbering{arabic}

\newcounter{stepnum}

%% Comments

\usepackage{color}


\title{%
    Requirements Specification\\
    \large Artificial Facial Aging through Deep Neural Networks}
\author{Jessica de Leeuw, Sam Cymbaluk,\\
    Jeff Gibson, and Fanping Jiang}

\begin {document}

\maketitle


\newpage

\section {Domain}
Authors: Jessica de Leeuw and Jeff Gibson\\

\medskip

\noindent The current method for artificially aging a face in missing persons cases is flawed. The ability to produce an accurate projection with a single picture of the missing person is a powerful solution to this problem. The goal is to implement an algorithm that successfully produces a realistic interpretation of the individuals aged face. \\

Stakeholders would be parties such as the government, and other law enforcement services. The generation of aged faces is useful for tasks such as searching for missing persons. The Canadian government currently employs forensic artists to manually create projections of a missing person at a specific age. Images are generated through drawing, and a significant amount of information (such as pictures of family members) is required about the missing person. The expectation would be that a single image is required as input, and a fairly accurate representation of that person aged a certain number of years would be outputted.\\

The main entities which the domain is comprised of are machine learning, big data, and the artificial aging of human faces.\\

A machine learning approach will be used to perform the process of artificial facial aging. To generate accurately aged faces, a conditional Generative Adversarial Neural Network (cGAN) must be trained on big data. The machine learning domain is greatly affected by this project, as it will be the first project publicly available that performs artificial aging. On a similar note, if this project can generate sufficiently accurate images, it will drastically improve upon current techniques. This would make it a possibility that this technique could become the industry standard for artificial aging of human faces.\\

\section{Machine Learning Training}
Author: Sam Cymbaluk\\

\noindent Component Description

\subsection{Functional Requirements}

Text

\subsection{Non-Functional Requirements}

\subsubsection {Reliability}

Text

\subsubsection {Accuracy of Results}

Text

\subsubsection {Performance}

Text

\subsubsection {Human-Computer Interface Issues}

Text

\subsubsection {Physical Constraints}

Text

\subsubsection {Portability Issues}

Text

\subsection{Requirements for the Development & Maintenance Process}

\subsubsection {Quality Control Procedures}

Text

\subsubsection {Priorities of the Required Functions}

Text

\subsubsection {Likely Changes to System Maintenance Procedures}

Text

\section{Data Storage}
Author: Jeff Gibson\\

\noindent Component Description

The data storage component is concerned with how data will be stored and indexed throughout the project.

\subsection{Functional Requirements}

\subsubsection{Data Processing} \label{data processing}

Only "clean" data from the datasets will be uploaded to the cloud for training. "Clean" data
is defined as members that meet the following criteria:\\

\begin{enumerate}
    \item Image must be of size 150px X 150px or greater
    \item Isolated face must occupy at least 50\% of the total image size
\end{enumerate}

\subsubsection{Cloud Storage}

Data will be in a GCP cloud storage bucket and indexed with an SQL database so that
it is easily accessible for training. Along with the data, trained versions of the algorithm
will be stored on the cloud. This means the database will have the functionality to upload files,
specifically trained versions of the algorithm.

\subsection{Non-Functional Requirements}

\subsubsection {Reliability}

It is very important that data is not lost, and does not become corrupt. To ensure this the
database will follow the acid principle.

\subsubsection {Performance}

As training is already a very long process it is important that the database implements SQL best
practices to ensure acceptable lookup times.

\subsubsection {Physical Constraints}

To work around the large storage space requirements, the project will utilize a GCP cloud bucket
storage space.

\subsubsection {Portability Issues}

This is not a concern. Since it is a database, there is no reason to move it.

\subsection{Requirements for the Development & Maintenance Process}


\subsubsection {Quality Control Procedures}

as stated above, it is important that the database implements SQL best practices, as well as
following the acid principle.

\subsubsection {Priorities of the Required Functions}

TBD

\subsubsection {Likely Changes to System Maintenance Procedures}

TBD

\section{Front-end Interface}
Author: Fanping Jiang\\

\noindent Component Description

\subsection{Functional Requirements}

Text

\subsection{Non-Functional Requirements}

\subsubsection {Reliability}

Text

\subsubsection {Accuracy of Results}

Text

\subsubsection {Performance}

Text

\subsubsection {Human-Computer Interface Issues}

Text

\subsubsection {Physical Constraints}

Text

\subsubsection {Portability Issues}

Text

\subsection{Requirements for the Development & Maintenance Process}

\subsubsection {Quality Control Procedures}

Text

\subsubsection {Priorities of the Required Functions}

Text

\subsubsection {Likely Changes to System Maintenance Procedures}

Text

\section{Machine Learning Inference}
Author: Jessica de Leeuw\\

\noindent Component Description

\subsection{Functional Requirements}

Text

\subsection{Non-Functional Requirements}

\subsubsection {Reliability}

Text

\subsubsection {Accuracy of Results}

Text

\subsubsection {Performance}

Text

\subsubsection {Human-Computer Interface Issues}

Text

\subsubsection {Physical Constraints}

Text

\subsubsection {Portability Issues}

Text

\subsection{Requirements for the Development & Maintenance Process}

\subsubsection {Quality Control Procedures}

Text

\subsubsection {Priorities of the Required Functions}

Text

\subsubsection {Likely Changes to System Maintenance Procedures}

Text

\end{document}
