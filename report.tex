\documentclass[12pt]{article}

\usepackage{graphicx}
\usepackage{paralist}
\usepackage{amsfonts}
\usepackage{amsmath}
\usepackage{hhline}
\usepackage{booktabs}
\usepackage{multirow}
\usepackage{multicol}
\usepackage{url}

\oddsidemargin -10mm
\evensidemargin -10mm
\textwidth 160mm
\textheight 200mm
\renewcommand\baselinestretch{1.0}

\pagestyle {plain}
\pagenumbering{arabic}

\newcounter{stepnum}

%% Comments

\usepackage{color}


\title{%
    Requirements Specification\\
    \large Artificial Facial Aging through Deep Neural Networks}
\author{Jessica de Leeuw, Sam Cymbaluk,\\
    Jeff Gibson, and Fanping Jiang}

\begin {document}

\maketitle


\newpage

\section {Domain}

-Describe application domain (what is that?)\\
-Talk about goals that the implementation should meet. (Domain knowledge required for specification)\\
-Who are the stakeholders and what are their expectations?\\

Stakeholders would be parties such as the government, and other law enforcement services. The generation of aged faces is useful for tasks such as searching for missing persons. This is currently handled manually; images are generated through drawing, and a lot of information (such as pictures of family members) is required about the missing person. The expectation would be that a single image is required as input, and a fairly accurate representation of that person aged a certain number of years would be outputted.

\medskip
-Main entities that characterize the domain\\
-What are the relationships between the main entities\\
-How are they affected by the system being developed? (Not sure what this means)

\section {Functional Requirements}

\subsection* {Generic Template Module}

Stack(T)

\subsection* {Uses}

N/A

\subsection* {Syntax}

\subsubsection* {Exported Types}

\ {What should be written here?}

\subsubsection* {Exported Constants}

None

\subsubsection* {Exported Access Programs}

\begin{tabular}{| l | l | l | p{5cm} |}
\hline
\textbf{Routine name} & \textbf{In} & \textbf{Out} & \textbf{Exceptions}\\
\hline
new Stack & seq of T & Stack & none\\
\hline
push & T & Stack & none\\
\hline
pop & & Stack & out\_of\_range\\
\hline
top & & T & out\_of\_range\\
\hline
size & & $\mathbb{N}$ & \\
\hline
toSeq& & seq of T & \\
\hline
\end{tabular}

\subsection* {Semantics}

\subsubsection* {State Variables}

$S$:  {What is the type of the state variable?}

\subsubsection* {State Invariant}

None

\subsubsection* {Assumptions \& Design Decisions}

\begin{itemize}
\item The Stack(T) constructor is called for each object instance before any
  other access routine is called for that object.  The constructor can only be
  called once.
\item Though the toSeq() method violates the essential property of the stack
  object, since this could be achieved by calling top and pop many times, this
  method is provided as a convenience to the client. In fact, it increases the
  property of separation of concerns since this means that the client does not
  have to worry about details of building their own sequence from the sequence
  of pops.
\end{itemize}

\subsubsection* {Access Routine Semantics}

new Stack($s$):
\begin{itemize}
\item transition: $S := s$

\item output: $\mathit{out} := \mathit{self}$
\item exception: none
\end{itemize}

\noindent push($e$):
\begin{itemize}
\item output: $out := \text{new Stack}(S\ ||\ \langle e \rangle)$
\item exception: none
\end{itemize}

\noindent pop():
\begin{itemize}
\item output:  {What should go here?}

\item exception:  {What should go here?}

\end{itemize}

\noindent top():
\begin{itemize}
\item output: $\mathit{out} := S[|S| - 1]$

\item exception:  {What should go here?}

\end{itemize}

\noindent size():
\begin{itemize}
\item output:  {What should go here?}
\item exception: None
\end{itemize}

\noindent toSeq():
\begin{itemize}
\item output: $\mathit{out} := S$
\item exception: None

\end{itemize}

\subsection{Non-Functional Requirements}

\subsection{Requirements for the Development \& Maintenance Process}

\end {document}
